\documentclass[a4paper]{article}


\usepackage{amsmath}
\usepackage{ctex}
\usepackage{graphicx}
\usepackage{float}
\usepackage{framed}
\usepackage{xcolor}

\title{近代物理讲义}


\author{陈虹骏}

\begin{document}
	\maketitle
	
	\section{波函数}
	\begin{figure}[H]
		\centering
		\includegraphics[width=1.2\linewidth]{fig_ModernPhys/WaveFunction}
		\caption{2023培尖4}
		\label{fig:wavefunction}
	\end{figure}
	
	$$p=\frac{h}{ \lambda }, E=h\nu $$
	
	对波函数的要求:
	
	归一性、单值性、连续性、有限性.
	
	
	\section{黑体辐射}
	\begin{figure}[H]
		\centering
		\includegraphics[width=1.2\linewidth]{fig_ModernPhys/blackbody}
		\caption{22爱培优5-8}
		\label{fig:blackbody}
	\end{figure}
	可以了解一下类似模型,之前王建秋也出过类似题.
	
	
	\section{光谱}
	\begin{figure}[H]
		\centering
		\includegraphics[width=1.2\linewidth]{fig_ModernPhys/spectrum}
		\caption{22启航8-7}
		\label{fig:spectrum}
	\end{figure}
	
	补充:X射线相关
	\begin{figure}[H]
		\centering
		\includegraphics[width=1.2\linewidth]{fig_ModernPhys/XRay}
		\caption{原子发射X Ray原理}
		\label{fig:xray}
	\end{figure}
	
	Moseley found the relationship between the atomic number Z and frequency f of K and L series, which is, $$f_K=cR_\infty(1-\frac{1}{n^2})(Z-1)^2$$and$$f_L=cR_\infty(\frac{1}{4}-\frac{1}{m^2})(Z-7.4)^2$$ where $n=2,3,4...$ and $m=3,4,5...$.
	
	
	\section{核物理}
	\begin{figure}[H]
		\centering
		\includegraphics[width=1.2\linewidth]{fig_ModernPhys/decay}
		\caption{2023北斗3}
		\label{fig:decay}
	\end{figure}
	\colorbox{yellow}{分离变量法}
	
	另外,2023猿辅导2也可以练练.
	\section{时空图}
	2023量子7,题太长遂不贴了.
	
	\section{相对论运动学——事件相关}
	\begin{figure}[H]
		\centering
		\includegraphics[width=1.2\linewidth]{fig_ModernPhys/RelaEvent}
		\caption{2023兰阶2}
		\label{fig:relaevent}
	\end{figure}
	使用梳理事件的方法.
	\section{相对论运动学——速度相关}
	\begin{figure}[H]
		\centering
		\includegraphics[width=1.2\linewidth]{fig_ModernPhys/RelaVelocity}
		\caption{2023思博楷瑞2}
		\label{fig:relavelocity}
	\end{figure}
	
	\section{相对论光的效应}
	\begin{figure}[H]
		\centering
		\includegraphics[width=1.2\linewidth]{fig_ModernPhys/RelaOptics}
		\caption{2023量子5}
		\label{fig:relaoptics}
	\end{figure}
	
	\section{托马斯进动}
	\begin{figure}[H]
		\centering
		\includegraphics[width=1.2\linewidth]{fig_ModernPhys/RelaThomas}
		\caption{2023思博楷瑞4}
		\label{fig:relathomas}
	\end{figure}
	
	\section{玻尔氢原子模型的相对论修正}
	\begin{figure}[H]
		\centering
		\includegraphics[width=0.7\linewidth]{../Downloads/pdf2png/part2-4/part2-4-1}
		\caption{相对论修正-1}
		\label{fig:part2-4-1}
	\end{figure}
	\begin{figure}[H]
		\centering
		\includegraphics[width=0.7\linewidth]{../Downloads/pdf2png/part2-4/part2-4-2}
		\caption{相对论修正-2}
		\label{fig:part2-4-2}
	\end{figure}
	\begin{figure}[H]
		\centering
		\includegraphics[width=0.7\linewidth]{../Downloads/pdf2png/part2-4/part2-4-3}
		\caption{相对论修正-3}
		\label{fig:part2-4-3}
	\end{figure}
	\begin{figure}[H]
		\centering
		\includegraphics[width=0.7\linewidth]{../Downloads/pdf2png/part2-4/part2-4-4}
		\caption{相对论修正-4}
		\label{fig:part2-4-4}
	\end{figure}
	\begin{figure}[H]
		\centering
		\includegraphics[width=0.7\linewidth]{../Downloads/pdf2png/part2-4/part2-4-5}
		\caption{相对论修正-5}
		\label{fig:part2-4-5}
	\end{figure}
	
	\section{飞船与火箭}
	\begin{figure}[H]
		\centering
		\includegraphics[width=1.2\linewidth]{fig_ModernPhys/photonrocket}
		\caption{2023兰阶2-3}
		\label{fig:photonrocket}
	\end{figure}
	
	\section{卢瑟福散射}
	\begin{figure}[H]
		\centering
		\includegraphics[width=1.2\linewidth]{fig_ModernPhys/RutherfordScatter}
		\caption{22猿辅导3-8}
		\label{fig:rutherfordscatter}
	\end{figure}
	相对论比奈方程.
	
	\section{碰撞}
	\begin{framed}
		用两种方法推导康普顿散射公式.
	\end{framed}
		
	\section{阈能问题与不变质量}
	不变质量$M^2 c^4=E^2-p^2c^2$表示的是零动量系中系统的运动状态(以及动能……).
	\begin{figure}[H]
		\centering
		\includegraphics[width=1.2\linewidth]{fig_ModernPhys/collision}
		\caption{2022思博楷瑞3}
		\label{fig:collision}
	\end{figure}
	
	\section{电磁场变换}
	\begin{figure}[H]
		\centering
		\includegraphics[width=1.2\linewidth]{fig_ModernPhys/EMRela}
		\caption{22启航10-8}
		\label{fig:emrela}
	\end{figure}
	
	\section{电动力学}
	\begin{figure}[H]
		\centering
		\includegraphics[width=1.2\linewidth]{fig_ModernPhys/EMRela2}
		\caption{22猿辅导1-3}
		\label{fig:emrela2}
	\end{figure}
	
	另,徐前批注质量非常高的题
	\begin{figure}[H]
		\centering
		\includegraphics[width=1.2\linewidth]{fig_ModernPhys/EMRela3}
		\caption{XPhO1st-6}
		\label{fig:emrela3}
	\end{figure}
	
	\section{切伦科夫辐射}
	\begin{figure}[H]
		\centering
		\includegraphics[width=1.2\linewidth]{fig_ModernPhys/Chelenkov}
		\caption{2023北斗2}
		\label{fig:chelenkov}
	\end{figure}
\end{document}